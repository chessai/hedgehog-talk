\section{What is Property Testing?}
    
    \frame{\sectionpage}
   
   
    \begin{frame}[fragile]{The Problems With Testing}
        \begin{itemize}
            \item How expensive is breakage?
            \item Tooling (PL, build system, etc.)
            \item Types of testing, verification
            \item Unit Tests are Extremely common
        \end{itemize}
        \begin{minted}[linenos,fontsize=\footnotesize]{haskell}
        -- 'hspec' unit test example
        myAssertion :: Assertion
        myAssertion = myFunction exampleInput `shouldBe` expectedOutput
        \end{minted}
    \end{frame}

   \begin{frame}[fragile]{The Problems With Testing (Cont'd)}
       \begin{definition}{Unit Test}
          A test that checks whether the output of a function, for a single
          example, is equal to the expected output.
       \end{definition}
       
       But, there are problems!
       
       \begin{itemize}
           \item Coverage Problem: Coverage of all code paths
           \item Developer Cost: Costly w.r.t developer time
           \item Unenjoyable to write (bad for morale)
       \end{itemize}
   \end{frame}
   
    \begin{frame}{Property Testing As A Solution To The Coverage Problem}
        Property testing:
            \begin{itemize}
                \item Parametric in its inputs
                \item Based on declarative \textit{properties}, rather than manually-constructed inputs
                \item The inputs are generated, rather than supplied by humans
            \end{itemize}
        
        How do you generate these inputs? Depends:
            \begin{itemize}
                \item Exhaustive property tests
                \item Randomised property tests
                \item Exhaustive, specialised property tests
                \item Randomised, specialised property tests
            \end{itemize}
    \end{frame}

    \begin{frame}[fragile]{Example}
        \begin{minted}[linenos,fontsize=\footnotesize]{haskell}
        prop_reverse :: [Char] -> Bool
        prop_reverse str = reverse (reverse str) == str
        \end{minted}       
    \end{frame}

    \begin{frame}[fragile]{Generators}
        \begin{itemize}
            \item Randomised generators are a way to describe how you should generate some input to a property test
            \item Appropriate generators can be the difference between finding bugs, and not finding bugs
            \item Not all types are easy to generate (e.g.: Word8 vs String)
            \item Generators can often be constructed using combinators, allowing smarter construction of generators (e.g. filtering,sizing)
        \end{itemize} 
        
        Example where a bad generator could fail:
        \begin{minted}[linenos,fontsize=\footnotesize]{haskell}
        prop_isSmall :: [a] -> Bool
        prop_isSmall ls = length ls < 100
        \end{minted}
    \end{frame}
   
    \begin{frame}[fragile]{Shrinking}
        \begin{itemize}
            \item Property test failures result in counterexamples can be large
            \item This can make reasoning about the failure harder
            \item What if we could shrink counterexamples to manageable sizes?
        \end{itemize} 
        
        \begin{minted}[linenos,fontsize=\footnotesize]{haskell}
        prop_allLower :: [Char] -> Bool
        prop_allLower str = all isLower (map toLower str)
        
        -- Counterexample
        -- "As5Enu3au04"
        
        -- Shrunken counterexample
        -- "1"
        
        -- λ toLower '1'
        -- '1'
        \end{minted}
        
        Note that the shrunk counterexample "1" is not necessarily part of the
        original counterexample.
    \end{frame}
    
    \begin{frame}[fragile]{Shrinking With Invariants}
        \begin{itemize}
            \item Shrunk values must have been able to come from the generator that generated the unshrunk value
        \end{itemize}
        
        Consider:
        
        \(\forall n. (n > 6) \Rightarrow (n > 5 \land odd(n)) \)
       
        If shrinking does not obey the invariants of the generator:
        
        \begin{minted}[fontsize=\footnotesize]{haskell}
        -- "Failed: the following number larger than six does not pass the property test: 0" 
        \end{minted}
    \end{frame}


